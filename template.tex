\documentclass[aspectratio=169]{beamer}
\usepackage[brazil]{babel}
\usepackage[utf8]{inputenc}
\usepackage{xcolor}
\usepackage{graphicx}
\usepackage{multicol}
\usepackage{tikz}



% Descomente as linhas abaixo para ativar slides intermediários (início de Seção e Subseção)

% \newcommand{\interludeTitle}{Estamos aqui!}
% \AtBeginSection[] {
%     \frame{
% 	\frametitle{\interludeTitle}
%       \begin{multicols}{2}
%         \tableofcontents[
%             currentsection,
%             sectionstyle=show/shaded,
%             ]
% 	  \end{multicols}
%     }
% }

% \AtBeginSubsection[] {
% 	  \frame{
% 		\frametitle{\interludeTitle}
% 		\begin{multicols}{2}
%         \tableofcontents[
%             currentsubsection,
%             subsectionstyle=show/shaded,
%             ]
% 		\end{multicols}
% 	  }
% }



%%%%%%%%%%%%%%%%%%%%%%%%%%%%%%%%%%%%%%%%%%%%%%%%%%%%%%%%%%%%
%%%%%%%% Escolher aqui entre tema branco ou escuro %%%%%%%%%
\usetheme{branco} 
%%%%%%%%%%%%%%%%%%%%%%%%%%%%%%%%%%%%%%%%%%%%%%%%%%%%%%%%%%%%
%%%%%%%%%%%%%%%%%%%%%%%%%%%%%%%%%%%%%%%%%%%%%%%%%%%%%%%%%%%%

% Talk date
% Uncomment this to define a presentation date distinct from \today
% \def\mydate{20 Feb 2000}

% Preamble
\title[Título]{Título da Palestra\\em duas linhas}
\subtitle{Conferência}
\author[autor]{\texorpdfstring{Prof(a). Dr(a). Fulano(a) de Tal\\\url{email@de.ufpb.br}}{Fulano(a) de Tal}}

% Body
\begin{document}
    
    
    {
        \setbeamertemplate{footline}{}
        \setbeamertemplate{navigation symbols}{}
        \frame{
            \titlepage
        }
    }
    
    \setcounter{framenumber}{0}

    % TOC
    \frame{
        \frametitle{Sumário}
        % \begin{multicols}{2}
        \tableofcontents
        % \end{multicols}
    }



    \section{Introdução}

    \frame{
        \frametitle{Sobre o Departamento}
        \begin{block}{Fundação}
            O Curso de Bacharelado em Estatística foi criado no ano 2000, formando a primeira turma em 2004.1.
        \end{block}
        \begin{itemize}
            \item Um item
            \begin{itemize}
                \item Um sub-item
                \begin{itemize}
                    \item Um sub-sub-item
                \end{itemize}
            \end{itemize}
        \end{itemize}
    }

    
    \section{Seção}

    \frame{
        \frametitle{Generic slide}
        \begin{enumerate}
            \item Um item
            \begin{enumerate}[a.]
                \item Um sub-item
                \begin{enumerate}[i.]
                    \item Um sub-sub-item
                    \item Outro sub-sub-item
                \end{enumerate}
                \item Outro sub-item
            \end{enumerate}
            \item Outro item
        \end{enumerate}
    }
    \subsection{Subseção}

    \frame{
        \frametitle{Generic slide}
        \framesubtitle{A bit more information about this}
        %More content goes here
    }

    
    \subsection{Outra Subseção}

    \frame{
        \frametitle{Generic slide}
        \framesubtitle{A bit more information about this}
        %More content goes here
    }
    
    \section{Conclusões}
    
    \frame{
        \frametitle{Generic slide}
        \framesubtitle{A bit more information about this}
        %More content goes here
    }
    
    % \section{Agradecimentos}
    
    % \frame{
    %     \frametitle{Generic slide}
    %     \framesubtitle{A bit more information about this}
    %     %More content goes here
    % }

    \section{}
    \frame[plain]{\titlepage}
\end{document}

